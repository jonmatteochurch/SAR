\chapter{Conclusioni}\label{cap4}

Una corretta analisi e un'accurata previsione in un campione
caratterizzato da distribuzione spaziale, devono essere in grado di
sfruttare le informazioni provenienti dagli errori delle
osservazioni vicine.\\
 Il metodo SAR \� in grado di cogliere le
correlazioni spaziali tra le osservazioni, almeno tra le pi\�
vicine, e questo lo rende molto pi\� adeguato nella rappresentazione
del campione, infatti abbiamo visto nel capitolo \ref{cap3} quanto
l'indice di determinazione $R^2$ sia notevolmente
migliore di quello ottenuto con una regressione lineare classica.\\
Inoltre se si vuole svolgere un'analisi di previsione su individui
che sono spazialmente prossimi a quelli del campione di riferimento,
abbiamo visto come SAR ottenga risultati molto buoni sfruttando le
informazioni sulla variabile risposta degli individui del campione
iniziale.\\

Tuttavia per poter implementare l'analisi in tempi modesti, abbiamo
dovuto scegliere di lavorare con un campione di dimensioni
relativamente ridotte, a causa dell'elevato costo computazionale del
metodo.
