\chapter{Esempio su data set reale}\label{cap3}

In questo capitolo vedremo l'applicazione del modello di regressione
lineare (OLS) e di autoregrassione simultanea (SAR) su un data set
reale caratterizzato da distribuzione spaziale. Nel
Par.~\ref{sect3.1} vengono illustrati data set e modello, nel
Par.~\ref{sect3.2} sono riporati i risultati ottenuti con le 2
diverse analisi e nel Par.~\ref{sect3.3} tali risultati vengono
rappresentati su grafici comodi per la visualizzazione del
confronto, infine nel Par.~\ref{sect3.4} vengono utilizzati i
parametri stimati per ottere previsioni su un secondo data set, del
quale sono noti i valori della variabile risposta.

\section{Data set reale}\label{sect3.1}

Il data set a nostra disposizione contiene informazioni relative a
$20640$ individui, che rappresentano tutti i condomini della
California tratti da un censimento del $1990$. Per ogni condominio
si hanno a disposizione le seguenti osservazioni: numero di
occupanti, rendita, et\�, numero totale di vani, numero totale di
camere, numero di proprietari, valore,latitudine e longitudine.\\
Le variabili di interesse sono:
\begin{itemize}
\item[] variabile risposta $Y = \ln(\mbox{valore di un condominio})$
\item[] $X_1=  \mbox{rendita}$
\item[] $X_2= (\mbox{rendita})^2$
\item[] $X_3= (\mbox{rendita})^3$
\item[] $X_4= \ln(\mbox{et\�})$
\item[] $X_5= \ln(\mbox{totale vani / numero occupanti})$
\item[] $X_6= \ln(\mbox{totale camere / numero occupanti})$
\item[] $X_7= \ln(\mbox{numero occupanti / numero proprietari})$
\item[] $X_8= \ln(\mbox{numero proprietari})$
\end{itemize}

Il modello che si vuole analizzare \� quindi:

$$ Y = \beta_0 +
\beta_1X_1+\beta_2X_2+\beta_3X_3+\beta_4X_4+\beta_5X_5+\beta_6X_6+\beta_7X_7+\beta_8X_8$$\\

Le variabili latitudine e longitudine verranno utilizzate per
costruire la matrice dei pesi $\bW$ nel modello SAR.

\section{Coefficienti stimati per i 2 modelli}\label{sect3.2}


Per motivi di costo computazionale, si \� scelto di ridurre
notevolmente il campione a sole $n=1000$ aree di interesse. \\
Si \� fissato inoltre a $m=4$ il numero di elementi non nulli su
ogni riga di $\bW$. In questo modo si ritiene cio\� che l'errore
relativo ad ogni osservazione dipenda solo da altre $4$
osservazioni.

La Tabella \ref{tab:table} contiene i valori stimati dei
coefficienti $\bb$ per i $2$ modelli OLS e SAR, i valori di
$S^2=\frac{\beps'\beps}{n-r-1}$ come stimatore di $\sigma^2$ e
quelli dell'indice di determinazione $R^2 =
1-\frac{\beps'\beps}{\sum_{i=1}^{n}{(y_i-\bar{y}_i)}^2}$.\\

\begin{table}[htbp]
\caption{Stime con OLS e SAR}
 \label{tab:table}
\begin{center}
\begin{tabular}{|c|c|c|}
 \hline
\textbf{} & \textbf{OLS} & \textbf{SAR}\\
\hline
\textbf{$\beta_0$} & $11.439828159$ & $11.565929281$ \\
\hline
\textbf{$\beta_1$} & $0.119867563$ & $-0.035349870$ \\
\hline
\textbf{$\beta_2$} & $0.022258423$ & $0.025167169$ \\
\hline
\textbf{$\beta_3$} & $-0.001689108$ & $-0.001400044$ \\
\hline
\textbf{$\beta_4$} & $0.062130790$ & $-0.010339510$ \\
\hline
\textbf{$\beta_5$} & $-0.148146463$ & $0.231575036$ \\
\hline
\textbf{$\beta_6$} & $-0.038564672$ & $-0.255060934$ \\
\hline
\textbf{$\beta_7$} & $-0.638793568$ & $-0.234159068$ \\
\hline
\textbf{$\beta_8$} & $0.083856199$ & $0.049893059$ \\
\hline
\textbf{$\lambda$}&   & $0.71442$ \\
\hline
\textbf{$\sigma^2$}& $0.06612056$ & $0.03514641$ \\
\hline
\textbf{$R^2$}     & $0.6448312$ & $0.8112099$ \\
\hline
\end{tabular}
\end{center}

\end{table}

Si pu\� notare come il coefficiente di determinazione della SAR sia
molto migliore di quello di OLS, il che \� significativo di come il
modello SAR sia pi\� adeguato a rappresentare il campione.

\section{Rappresentazione grafica dei risultati}\label{sect3.3}

Poich\� i regressori sono $8$, la rappresentazione completa del
campione necessita di uno spazio a $9$ dimensioni. Si pu\� pensare
allora di proiettare i dati su $8$ piani $X_i \times Y$, in modo da
vedere come la variabile risposta vari lungo ogni regressore.\\
Se alle $Y_i$ note si aggiungono i valori fittati $\hat{Y}_i$ per
OLS e $\check{Y}_i$ per SAR, si pu\� vedere quanto i valori predetti
siano prossimi o meno a quelli noti. In Figura 3.1 sono riportati i
valori della variabile risposta (in rosso) e dei valori fittati in
corrispondenza delle osservazioni delle variabili $X_1$ e $X_4$, i
punti verdi corrispondono al modello OLS, quelli blu al SAR. Anche
se \� evidente come i valori fittati ottenuti con la SAR siano pi\�
prossimi ai valori reali di quelli ottenuti con la regressione
lineare, queste rappresentazioni non sono molto
efficaci.\\



\begin{figure} [h] {\setlength{\fboxsep}{2mm}
\fbox{\parbox{120mm}{%
\begin{center}
\placefig{fig_cap3/fig1}{11cm}
\end{center}

}}
\begin{center} \caption{Proiezioni su $X_1 \times Y$ e $X_4 \times Y$ di $Y$, $\hat{Y}$, $\check{Y}$ .}
\end{center} \label{fig3.1} }
\end{figure}

Abbiamo gi\� detto al Par.~\ref{sect2.1} che le componenti di $\bhY$
ottenute con OLS giacciono su l'iperpiano affine di
$\mathbb{R}^{r+1}$ di equazione

$$Y = \hat{\beta}_0 +\hat{\beta}_1 X_1 +\ldots+\hat{\beta}_r X_r$$

Sarebbe quindi interessante poter visualizzare nei piani i valori
fittati in modo da rappresentarne l'allineamento. A tal fine si pu\�
pensare di proiettare i dati sul generico piano $X_i \times Y$,
invece che ortogonalmente al piano stesso, parallelamente
all'iperpiano
generato dalle colonne di $\bZ$.\\

Consideriamo per semplicit\� il caso $\mathbb{R}^3$ in modo da poter
visualizzare lo spazio. In questo caso allora l'iperpiano OLS ha
equazione:

$$Y = \hat{\beta}_0 + \hat{\beta}_1 X_1 +\hat{\beta}_2 X_2$$
ed \� generato dai vettori ortogonali $(1,0,-\hat{\beta}_1), (0,
1,-\hat{\beta}_2)$ , quindi se si considera un generico punto $(
x_{1},x_{2},y)$, la sua proiezione su $X_1 \times Y$ parallelamente
a questo piano ha per coordinate $(x_{1},y-\hat{\beta}_2 x_{2})$,
mentre quella su $X_2 \times Y$ � $(x_{2}, y-\hat{\beta}_1 x_1)$. La
figura 3.2 chiarisce il processo.

\begin{figure} [h] {\setlength{\fboxsep}{2mm}
\fbox{\parbox{120mm}{%
\begin{center}
\placefig{fig_cap3/fig2}{11cm}
\end{center}

}}
\begin{center} \caption{Generico punto proiettato ortogonalmente al piani $X_1 \times Y$ e $X_2 \times Y$
 e poi parallelamente al piano generato con OLS.}
\end{center} \label{fig3.2} }
\end{figure}

Generalizzando al caso $\mathbb{R}^{r+1}$, la proiezione sul
generico piano $X_i \times Y$ parallelamente all'iperpiano OLS $Y =
\hat{\beta}_0+\hat{\beta}_1 X_1+\ldots+\hat{\beta}_r X_r$ del punto
$$(\mathbf{x},y)=(x_{1},x_{2},\ldots,x_{r},y)$$
ha per coordinate
$$(x_{i},y - \mathbf{x} \cdot \bhb^{*})$$
dove $\bhb^{*}$ \� uguale al vettore $\bhb$ tranne che per la
$i$-esima coordinata che \� nulla.

Proiettiamo in questo modo sia i valori reali assunti dalla
variabile risposta che quelli fittati ottenuti dai 2 modelli.
Ovviamente le proiezioni dei $\bhY$ sono allineate lungo una retta.
In figura 3.3 e 3.4 sono riportati i grafici delle proiezioni su
tutti i piani, in rosso le proiezioni dei dati reali, in verde
quelle dei valori ottenuti con OLS e in blu quelli di SAR.


Questi grafici mettono in evidenza come i valori fittati stimati da
SAR sono molto pi\� simili a quelli reali che non quelli trovati da
OLS.
\newpage

\begin{center}
\begin{figure} [h] {\setlength{\fboxsep}{2mm}
\fbox{\parbox{120mm}{%

\placefig{fig_cap3/fig3}{7.2cm}


}}

 \caption{Proiezione dei dati reali e fittati con i 2 modelli parallelamente all'iperpiano OLS}
 \label{fig3.3} }
\end{figure}
\end{center}

\begin{center}
\begin{figure} [h] {\setlength{\fboxsep}{2mm}
\fbox{\parbox{120mm}{%

\placefig{fig_cap3/fig4}{7.2cm}


}}
 \caption{Proiezione dei dati reali e fittati con i 2 modelli parallelamente all'iperpiano OLS}
 \label{fig3.4} }

\end{figure}
\end{center}

\newpage

\section{Previsioni}\label{sect3.4}

Inizialmente si \� specificato che si dispone in realt\� di molte
pi\� osservazioni di quante utilizzate per le analisi. Si \� scelto
pertanto di utilizzare $100$ delle osservazioni restanti per
svolgere delle analisi di previsione e poi confrontare i valori
fittati ottenuti con quelli reali.\\
Sia $\mathbf{Z}^0 =
[\mathbb{I}_{n_{0}}|\mathbf{X}_{1}^{0}|\ldots|\mathbf{X}_{r}^{0}]$
il nuovo campione di osservazioni sui regressori composto da $n_0 =
100$ individui.

I valori fittati della variabile risposta ottenuti con OLS sono
dunque:

$$\bhY^0 =  \bhb \mathbf{Z}^0$$

Per quanto riguarda la SAR invece:

$$\cY^0 =  \cb \mathbf{Z}^0 + \check{\lambda}\bW(\bY^0 - \cb \mathbf{Z}^0)$$

Ora per\� i valori $\mathbf{Y}^0$ non sono noti, ma sono proprio le
quantit\� da stimare. L'unica stima che si pu\� ottenere allora \�:

$$\cY^0 =  \cb \mathbf{Z}^0 $$

Ma come prevedibile i valori fittati cos\� ottenuti sono molto
distanti da quelli reali, e quindi le previsioni ottenute con il
metodo SAR sono abbastanza deludenti, sicuramente peggiori di quelle
ottenute con OLS.

Si pu\� ovviare per\� a questo problema pensando di utilizzare
l'eventuale correlazione tra le osservazioni del nuovo campione di
dati $\bZ^0 $ e quelle del campione originale $\bZ$ sulle quali \�
stata svolta la regressione.

In effetti abbiamo visto al Par.~\ref{sect2.4} che per ogni
osservazione $i$-esima, solo $m$ delle altre osservazioni
influenzano la stima di $\check{Y}_i$, e di certo non se stessa per
la propriet\� di non autopredittibilit\� della matrice $\bW$.
Possiamo supporre allora che le $m$ osservazioni da cui dipende il
generico individuo del nuovo campione non appartengano a tale nuovo
campione, ma al campione iniziale, del quale sono noti i valori
reali della variabile risposta.

Il nuovo modello di previsione pu\� allora essere espresso come:

$$\cY^{0} =  \cb \mathbf{Z}^0 + \check{\lambda}\bW^{0}(\bY - \cb \mathbf{Z})$$

$\bW^0 \in \mathbb{R}^{n_0\times n_0}$ \� la nuova matrice dei pesi,
ed \� tale che $w_{ij}^0 \neq 0$ se il $j$-esimo individuo di $\bZ$
\� uno degli $m$ pi\� vicini all'$i$-esimo individuo di $\bZ^0$.

 Questo modello di previsione ha senso se gli
individui del nuovo campione sono correlati con quelli del campione
iniziale, ovvero se ci sono almeno $m$ punti di $\bZ$ che sono
abbastanza vicini in termini di distanza spaziale da i punti di
$\bZ^0$.

Disponendo dei veri valori della variabile risposta, siamo in grado
di confrontare questi con quelli previsti ottenuti con i $2$
modelli. I risultati ottenuti sono rappresentati in figura 3.5, con
la solita convenzione rosso per i valori reali, verde per OLS e blu
per SAR.

\begin{center}
\begin{figure} [h] {\setlength{\fboxsep}{2mm}
\fbox{\parbox{120mm}{%

\placefig{fig_cap3/fig5}{11cm}

}}
 \caption{Proiezione delle previsioni su
$X_1 \times Y$ e $X_4 \times Y$ ortogonalmente ai piani e
parallelamente all'iperpiano OLS}
 \label{fig3.5} }
\end{figure}
\end{center}


Dai grafici sembra che le previsioni ottenute con SAR non siano
buone.

A questo punto andiamo a vedere se effettivamente le correlazioni
che sono state supposte tra il nuovo campione e quello iniziale
esistono, ovvero se i $2$ campioni sono vicini in termini di
coordinate spaziali. In Figura 3.6 sono raffigurate le disposizioni
spaziali del campione iniziale (in nero) e del secondo campione (in
arancio).
\newpage

\begin{center}
\begin{figure} [h] {\setlength{\fboxsep}{2mm}
\fbox{\parbox{120mm}{%

\placefig{fig_cap3/fig6}{7cm}

}}
 \caption{Distribuzione spaziale dei $2$ campioni}
 \label{fig3.6} }
\end{figure}
\end{center}

\`{E} evidente che i punti del secondo campione sono quasi tutti
molto distanti dal primo, il che spiega perch\� le previsioni
ottenute siano cos\� scadenti. Ci sono per\� $22$ punti del nuovo
campione che cadono in prossimit\� del primo (Figura 3.7).

\begin{center}
\begin{figure} [h] {\setlength{\fboxsep}{2mm}
\fbox{\parbox{120mm}{%

\placefig{fig_cap3/fig7}{7cm}

}}
 \caption{Distribuzione spaziale dei $2$ campioni}
 \label{fig3.7} }
\end{figure}
\end{center}

Proviamo a vedere se per queste 22 osservazioni la SAR fallisce o
fornisce delle stime migliori di OLS. In figura 3.8 sono riportati i
valori stimati di $Y$ con i 2 modelli e quelli reali, sui vari piani
descritti precedentemente.




\begin{center}
\begin{figure} [h] {\setlength{\fboxsep}{2mm}
\fbox{\parbox{120mm}{%

\placefig{fig_cap3/fig8}{11cm}

}}
 \caption{Proiezione delle previsioni ridotte su
$X_1 \times Y$ e $X_4 \times Y$ ortogonalmente ai piani e
parallelamente all'iperpiano OLS}
 \label{fig3.8} }
\end{figure}
\end{center}

Per questi 22 individui la SAR ottiene risultati di previsione molto
soddisfacenti, migliori di OLS.

Tutto ci\� \� in perfetto accordo con il modello che prevede che ci
sia correlazione solamente tra osservazioni vicine nello spazio.
