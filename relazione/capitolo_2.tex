\chapter{Il modello di autoregressione simultanea SAR}\label{cap2}


Nel Par.~\ref{sect2.1} viene brevemente trattato il modello di
regressione lineare classico (OLS) e i risultati ad esso correlati.
Nel Par.~\ref{sect2.2} viene descritto il modello di autoregressione
spaziale per una variabile risposta i cui valori sono tra loro
autocorrelati. Quindi nel Par.~\ref{sect2.3} viene presentato il
modello SAR (Simultaneous AutoRegression) per l'errore di
regressione e la relativa funzione di massima verosomiglianza. Tale
funzione dipende dalla matrice spaziale dei pesi $\bW$, la quale
viene trattata nel Par.~\ref{sect2.4}.


\section{Regressione Lineare (OLS)}\label{sect2.1}

 Siano $X_1, X_2, \ldots  , X_r$ $r$ variabili
(predittori) ritenute correlate con una variabile $Y$ (risposta).
Allora il modello di regressione lineare assume la forma
\[Y = \beta_0 + \beta_1 X_1 + \ldots+\beta_r X_r + \varepsilon \]

Dove $\varepsilon$ \� l'errore di regressione ed \� una variabile
aleatoria indipendente dai predittori e tale che $ \mathsf{E} (
\varepsilon) = 0 $  e
$ \mathsf{Var} ( \varepsilon ) = \sigma^2 $.\\

 Date $n$ osservazioni indipendenti della variabile risposta e
i corrispondenti valori dei predittori, il modello completo diventa
\[ \bY = \bZ \bb +\beps \]

\noindent
dove\\
$\bY \in \mathbb{R}^n $ vettore delle realizzazioni di $Y$ \\
$\bZ=
[\mathbb{I}_{n\times1}|\mathbf{X_1}|\ldots|\mathbf{X_r}] \in \mathbb{R}^{n\times(r+1)}$ matrice disegno \\
$\mathbf{X_i} \in \mathbb{R}^n $ vettore delle realizzazioni di $X_i$ \\
$\beps \in \mathbb{R}^n$ vettore degli errori tale che $ \mathsf{E}
( \beps ) = \mathbf{0} $  e
$ \mathsf{Cov} ( \beps ) = \sigma^2 \mathbb{I}_n$\\

$\bb$ e $\sigma^2$ sono i parametri del modello OLS. Con il metodo
dei minimi \linebreak
 quadrati si ottiene una stima di $\bb$ minimizzando
lo scarto quadratico medio $S(\mathbf{b})=
(\bY-\bZ\bb)'(\bY-\bZ\bb)$:

\[\bhb = \mathsf{argmin}_{\mathbf{b}\in\mathbb
R^{r+1}}S(\mathbf{b})\]


In seguito vengono riportati alcuni risultati noti sulla regressione
lineare (si omettono le dimostrazioni).\\

\textbf{Teorema 1:} \textit{ Sia $\bZ$ di rango pieno $r+1\leq n$, allora:}\\
\begin{itemize}
\item[ ]$\bhb = (\bZ'\bZ)^{-1}\bZ'\bY$
\item[ ]$\bhY=\bZ\bhb$
\item[ ]$\bheps=\bY-\bhY$\\
\end{itemize}

\textbf{Teorema 2:} \textit{ Sia $\bZ$ di rango pieno $r+1\leq n$, allora:}\\
\begin{itemize}
\item[ ]$ \mathsf{E} ( \bhb ) = \bb$
\item[ ]$\mathsf{Cov}(\bhb) =\sigma^2(\bZ'\bZ)�{-1}$
\item[ ]$ \mathsf{E} ( \bheps ) = \mathbf{0}$
\item[ ]$\mathsf{Cov} (\bheps)
=\sigma^2(\mathbb{I}_n-\mathbf{H})$ dove
$\mathbf{H}=\bZ(\bZ'\bZ)^{-1}\bZ'$
\item[ ]$\mathsf{E}(\bheps'\bheps)= \sigma^2(n-r-1))$\\
\end{itemize}


Si definisce indice di determinazione:
$$ R^2 =
1-\frac{\bheps'\bheps}{\sum_{i=1}^n(y_i-\bar{y})^2} $$  $R^2$ \� un
indice di quanto il modello spieghi adeguatamente i
dati.\\

\textbf{Corollario 1:}\\
$$S^2 =
\frac{\bheps'\bheps}{n-(r+1)}$$
\� uno stimatore non distorto di $\sigma^2$.\\

\textbf{Teorema 3:} \textit{ Sia $\mathrm{rank}(\bZ) = r+1 \leq n$
ed inoltre
 $\beps \sim \emph{N}_n(\mathbf{0},\sigma^2\mathbb{I}_{n} )$, allora:}\\
 \begin{itemize}
\item[ ] $\bhb$ e $\hat\sigma^2 = \bheps'\bheps/n$ sono stimatori di
massima verosomiglianza per $\bb$ e $\sigma^2$
\item[ ]$\bhb \sim
\emph{N}_{r+1}(\bb,\sigma^2(\bZ'\bZ)^{-1})$
\item[ ] $\bheps\sim
\emph{N}_n(\mathbf{0},\sigma^2(\mathbb{I}_n-\mathbf{H}))$
\item[ ] $\bheps$ e $\bhb$ sono tra loro ortogonali
\item[ ] $\hat\sigma^2 \sim \sigma^2 \chi(n-r-1)$\\
 \end{itemize}


\textbf{Corollario 2:} \\
$$\frac{(\bhb-\bb)'(\bZ'\bZ)(\bhb-\bb)}{S^2} \sim
(r+1)F(r+1,n-r-1)$$

\newpage
Da un punto di vista geometrico, i valori fittati trovati con OLS
$$\hat{Y}_i = \hat\beta_0+\hat\beta_1 X_1+\ldots+\hat\beta_r X_r$$
giacciono sull'iperpiano di $\mathbb{R}^{r+2}$ generato dalle
colonne di $\bZ$.

\section{Autoregressione spaziale}\label{sect2.2}

Quando i valori assunti $\bY = (y_1, y_2,\ldots, y_n)'$ da una
variabile $Y$ sono in alcuni punti fortemente correlati ai valori
che li precedono e seguono (autocorrelazione spaziale), $$y_t = f
(y_{t-1}, y_{t-2},\ldots, y_{t-p}, e_t ), \;\;\;\; \forall \;\; p +
1 \leq t \leq n$$ si adotta un modello di autoregressione spaziale,
che pu\� essere espresso come:
$$y_t = \beta_0 +\sum_{i=1}^{p}  \beta_i y_{t-i} + e_t , \;\;\;\;
\forall \;\; p + 1 \leq
t \leq n$$ dove:\\
$\beta_i$ sono i coeffcienti di autoregressione \\
$p \in \{1,\ldots,n$\} \`{e} l'ordine di autoregressione\\
$e_t$ \� il $t$-esimo errore.\\

Ponendo\\
$\widetilde{\bY} = (y_{p+1},y_{p+2},\ldots,y_p)'$\\
$\bb = (\beta_0,\beta_1,\ldots,\beta_p)'$\\
$\be= (e_{p+1},e_{p+2},\dots,e_n)' $\\
$ { \mathbf{X}} = \left(
\begin{array}{ccccc}
 1      & y_{p}   & y_{p-1} & \ldots &  y_1 \\
 1      & y_{p+1} & y_{p}   & \ldots &  y_2 \\
 \vdots & \vdots  & \vdots  &        &\vdots\\
 1      & y_{n-1} & y_{n-2} & \ldots &y_{n-p} \\
\end{array}
\right)\\ $
 il modello pu\`{o} essere espresso come per la regressione lineare
 nella forma
$$ \widetilde{\bY} = \mathbf{X}\bb + \be\\$$
ove evidentemente la matrice disegno $\mathbf{X}$ ha un significato
completamente diverso, poich\� in questo caso non ci sono
regressori, l'unica variabile \� $Y$, e i valori che assume su ogni
osservazione dipendono solo dai valori che la stessa assume sulle
altre osservazioni.

Per mettere in evidenza proprio la dipendenza del vettore $\bY$ solo
da se stesso, una forma pi\� generale del modello \�:

$$\bY = \lambda \mathbf{W}\bY + \mathbf{e}\\$$
dove $0 \leq \lambda < 1$ \� un parametro di autoregressione
spaziale e $\bW = w_{ij}$ \� una matrice di pesi
che soddisfa la condizione di non auto-predittibilit\�:\\
$$w_{ii}=0 \;\;\; \forall i \in\{1,\ldots,n\}$$
e di normalizzazione per righe:
$$\sum_{j=1}^{n}w_{ij} = 1\;\;\; \forall i \in\{1,\ldots,n\}\\$$




\section{Autoregressione simultanea SAR}\label{sect2.3}
Quando si trattano osservazioni distribuite spazialmente, se si
sceglie di utilizzare un modello di regressione lineare del tipo
$$\bY = \bZ\bb +\beps$$ si rischia di perdere gran parte del potere
predittivo del modello stesso poich\� si ignora la
correlazione spaziale dei dati.\\
 In particolare in questi casi l'errore di regressione
$\varepsilon$ \� spazialmente autocorrelato, quindi viene violata
l'ipotesi fondamentale di indipendenza tra gli errori sulle varie osservazioni. \\

Si pu\� pensare allora di adottare per $\varepsilon$ il modello di
autoregressione spaziale:
$$ \beps = \lambda \bW\beps + \be$$
e di sostituirlo nell'espressione della regressione lineare, si
ottiene cos\� il modello di autoregressione simultanea (SAR):
\begin{eqnarray*}
\bY &=& \bZ\bb + \lambda \bW \beps + \be\\
    &=& \bZ\bb +\lambda\bW(\bY-\bZ\bb) +\be
\end{eqnarray*}
In pratica SAR corregge la forma classica di regressione aggiungendo
una media pesata delle deviazioni $\bY - \bZ\bb$ sulle
osservazioni vicine.\\
\begin{itemize}
\item[*] La matrice dei pesi $\bW$ gode delle propriet\� descritte al Par.
\ref{sect2.2}:

$$w_{ii}=0, \;\;\; \forall i \in\{1,\ldots,n\}$$
$$\sum_{j=1}^{n}w_{ij} = 1\;\;\; \forall i \in\{1,\ldots,n\}\\$$
Un valore non nullo del generico $w_{ij}$ significa che la $j$-esima
osservazione verr\� usata per correggere la previsione
sull'$i$-esimo individuo, ci\� equivale a ritenere che gli errori
sulle $2$ osservazioni $i$ e $j$ siano correlati. Poich\� si stanno
trattando individui spazialmente correlati, i valori non nulli di
$\bW$
si trovano in corrispondenza di individui tra loro vicini nello spazio.\\

\item[*] $\lambda$ \� il parametro di autoregressione dell'errore e deve
essere tale che
$$ 0 \leq \lambda < 1$$

\item[*]Infine deve risultare
$$ \be \sim  \emph{N}_n(\mathbf{0},\sigma^2 \mathbb{I}_n)$$\\

\end{itemize}

 $\lambda, \sigma^2$ e $\bb$ sono i parametri del modello SAR e
vengono stimati massimizzando il logaritmo della funzione di massima
verosomiglianza:

$$L(\lambda,\sigma^2,\bb)= \frac{1}{2}\ln|\bB| -\frac{1}{2}[n\ln(2\pi\sigma^2) +
\sigma^{-2}(\bY-\bZ\bb)'\bB(\bY-\bZ\bb)]$$
dove $\bB = (\mathbb{I}_n-\lambda\bW)'(\mathbb{I}_n-\lambda\bW)$.\\
Affinch\� la somma degli scarti quadratici dell'errore,
$(\bY-\bZ\bb)'\bB(\bY-\bZ\bb)$ sia strettamente positiva, \�
necessario che $\bB $ sia definita positiva. Per definizione di
$\bW$ ed essendo $0 \leq \lambda <1$, risulta $0<| \bB| \leq 1$, e
quindi $-\infty<\ln|\bB|\leq 0$.

Supposta l'esistenza degli stimatori di massima verosomiglianza
$\check{\lambda}, \check{\sigma}^2$ e $\cb$, \� allora possibile
predire $\bY$ tramite
$$ \cY = \bZ\cb + \check{\lambda}
\bW(\bY-\bZ\cb)$$ e stimare l'errore
$$\ce=\bY-\cY$$

Purtroppo non si riesce ad ottenere alcuna forma analitica a priori
dei coefficienti o dei valori fittati, n\� delle loro distribuzioni,
pertanto non \� possibile svolgere test statistici o valutare
intervalli di confidenza per i parametri stimati (che invece \�
possibile per la regressione lineare se l'errore ha distribuzione
normale). Possiamo solo utilizzare i coefficienti stimati per fare
previsione su nuovi campioni.


\section{Matrice dei pesi $\bW$}\label{sect2.4}

 Per ogni individuo del campione, sono noti i
valori della variabile risposta $Y$, dei predittori $X_1,\ldots,X_r$
e le coordinate spaziali in termini di latitudine e longitudine,
tramite le quali \� possibile costruire la matrice dei pesi $\bW$.
Le componenti di $\bW$ sono infatti funzione della vicinanza
spaziale tra i vari individui.


Sia $d_{ij} =
\sqrt{{[(\mbox{latitudine})_i-(\mbox{latitudine})_j]}^2+{[(\mbox{longitudine})_i-(\mbox{longitudine})_j]}^2}$
la distanza Euclidea tra ogni coppia di individui $i $ e $j$, sia
inoltre $d_{max\,i}$ la distanza tra l'osservazione $i$ e il suo
$m$-esimo punto pi\� vicino. Si pone allora\\
\begin{eqnarray*}
 \widetilde{w}_{ij} &=&1 \mbox{ se } d_{ij} \leq d_{max\,i}\\
 \widetilde{w}_{ij}&=&0 \mbox{ altrimenti.}\\
 \end{eqnarray*}


 $\widetilde{\mathbf{W}}=(\widetilde{w}_{ij})$ non \� ancora la matrice dei pesi poich\� deve
 essere normalizzata per righe, e deve avere valori nulli sulla diagonale, quindi:\\
 \begin{eqnarray*}
 w_{ij} &=& \frac{\widetilde{w}_{ij}}{\sum_{j=1,\,i\neq j}^{n}\widetilde{w}_{ij}}\\
 w_{ii} &=& 0 \;\;\; \forall i \in\{1,\ldots,n\}\\
\end{eqnarray*}

In questo modo per ogni previsione $\check{Y}_i$, le $m$
osservazioni pi\� vicine ad essa (eccetto lei stessa) hanno un peso
pari a $\frac{\check\lambda}{m}$ nella sua valutazione, mentre le
altre non contano, ovvero:
$$ \check{Y}_i = \check\beta_0 + \check\beta_1 x_{i1} + \ldots + \check\beta_r
x_{ir} +
\frac{\check\lambda}{m}\sum_{k=1}^{m}(Y_k^{*}-\check\beta_0-
\check\beta_1 x_{k1}^{*} - \ldots - \check\beta_r x_{kr}^{*})$$\\
dove $^{*}$ indica che l'individuo $k$ \� uno degli $m$ del campione
pi\� vicini all'osservazione $i$ in termini di distanza spaziale.\\

Se si sceglie $m=n$ la matrice $\bW $ \� piena, e questo significa
che esistono $n^2$ potenziali correlazioni. Poich\� per stimare i
parametri della SAR \� necessario valutare il determinante di tale
matrice, sono necessarie $n^3$ operazioni, quindi il costo
computazionale diventa veramente molto alto se il numero di
osservazioni \� elevato.

Fortunatamente l'autocorrelazione spaziale degli errori di norma va
diminuendo all'aumentare della distanza, ecco perch\� si sceglie di
troncare l'influenza delle osservazioni ai soli $m$ punti pi\�
vicini, con $m$ non molto elevato. Scegliere $m$ piccolo \� quindi
in accordo con il modello teorico, ed inoltre fa si che la matrice
dei pesi non sia piena ma abbia solo un numero ridotto di elementi
non nulli.

Da un punto di vista numerico \� conveniente usare un matrice in
forma sparsa, con questa tecnica vengono memorizzati solo gli
elementi non nulli della matrice e si evita di calcolare ogni volta
un'operazione con un elemento nullo. In questo modo diminuisce lo
spazio di memoria necessario al calcolatore e si accelerano i tempi
computazionali.

Non entriamo nei dettagli computazionali, ci limitiamo a dire che il
modello SAR \� implementato nella \textit{spdep} libreria del
software R da noi utilizzato per svolgere l'analisi, richiede che
vengano fornite le informazioni: valori assunti dalla variabile
risposta e dai predittori sulle osservazioni, matrice $\bW$ in forma
sparsa. Anche per la costruzione di $\bW$ si pu\� far riferimento
alla stessa libreria, basta fornire una matrice contenente
latitudine e longitudine delle varie osservazioni e il valore $m$ di
vicini influenti, per maggiori chiarimenti il codice \� riportato in
Appendice A.
