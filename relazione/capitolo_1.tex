\chapter{Introduzione}\label{cap1}

La regressione lineare \� forse una delle tecniche pi\� utilizzate
in ambito statistico. Quando si trattano osservazioni distribuite
spazialmente, se si sceglie di utilizzare un modello di regressione
lineare, si rischia di perdere gran parte del potere predittivo del
modello stesso poich\� si ignora la correlazione spaziale dei dati.
In particolare, l'errore di regressione $\varepsilon$ \�
spazialmente autocorrelato, quindi non \� verificata l'ipotesi
fondamentale del modello di indipendenza tra
gli errori sulle varie osservazioni del campione.\\

Un'alternativa alla regressione lineare \� il modello di
autoregressione simultanea (SAR). In questo modello per l'errore di
regressione viene adottato un sotto modello di autocorrelazione
spaziale, che permette di tenere conto del fatto che osservazioni
vicine nello spazio sono tra loro correlate. In questo modo SAR
risulta essere molto pi\� efficace della regressione lineare non
solo nel rappresentare il campione su cui si \� costruita l'analisi,
ma anche nella previsione su osservazioni future.


Purtroppo questa tecnica necessita la conoscenza di tutte le
relazioni tra le varie osservazioni del campione, che \�
rappresentabile con una matrice delle distanze tra i punti di
dimensione $n\times n$, dove $n$ \� il numero di osservazioni a
disposizione. La stima dei parametri del modello richiede il calcolo
del determinante di tale matrice, ovvero la risoluzione di $n^3$
operazioni, che per un campione numeroso risulta essere un costo
computazionale molto elevato.

Fortunatamente, la correlazione spaziale di norma va diminuendo con
l'aumentare della distanza spaziale tra i punti. Se si sceglie di
fissare un numero limite $m$ di osservazioni influenti per ogni
individuo e di annullare l'effetto quindi delle $n-m$ osservazioni
pi\� lontane, il numero di relazioni necessarie per la stima dei
parametri diminuisce notevolmente (elementi non nulli della matrice
delle distanze). Allora diminuisce il costo computazionale e si
accelerano i tempi di calcolo.\\
Anche se il numero di elementi non nulli della matrice delle
distanze \� sceso a $n\times m$, per campioni numerosi il metodo
richiede comunque un costo elevato.\\

Nel capitolo \ref{cap2} dopo una breve trattazione dei concetti
fondamentali della regressione lineare, viene descritto il modello
di autocorrelazione spaziale che verr\� applicato all'errore, e il
modello definitivo di autoregressione simultanea (SAR).

Nel capitolo \ref{cap3} viene riportata l'applicazione del modello
ad un caso reale ed il relativo confronto con i risultati della
regressione lineare.

Infine nel capitolo \ref{cap4} sono illustrate le conclusioni
ottenute dall'applicazione del modello SAR.
