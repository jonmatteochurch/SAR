\chapter[Appendice A]{Appendice}\label{Appendice}


CODICE \textbf{SAR}:\\

\begin{verbatim}

######################################
## PROGETTO DI STATISTICA APPLICATA ##
 ########### A.A. 2007-2008###########
## Dalila Vescovi Jon Matteo Church ##
######################################

## SAR: Spatial AutoRegresion ##

## data set:

setwd('C:/Documents and Settings/Dalila
Vescovi/Desktop/progetto_1-7-08/DALILA')

dati.comp <- read.table('dataset.txt', header = T) dati <-
dati.comp[1:1000,]

attach(dati)

Y <- log(median_house_value)
X1 <- median_income
X2 <- median_income^2
X3 <- median_income^3
X4 <- log(housing_median_age)
X5 <- log(total_rooms/population)
X6 <- log(total_bedrooms/population)
X7 <- log(population/households)
X8 <- log(households)
lat <- latitude
long <- longitude
detach(dati)
n <- length(Y)
p<-8
X <- cbind(X1,X2,X3,X4,X5,X6,X7,X8)
data <-data.frame(Y,X1,X2,X3,X4,X5,X6,X7,X8)
Z <-cbind(rep(1,n),X1,X2,X3,X4,X5,X6,X7,X8)
####################################################### ## Matrice
dei pesi W ## distanze con latitudine e longitudine:

library(spdep)

d <- knearneigh(cbind(lat,long), k=4)
## d matrice n x 4,
## la riga i_esima contiene le etichette dei 4 punti pi� vicino al
## punto i sulla base della distanza euclidea calcolata tramite lat
## e long.

W. <- knn2nb(d)

W <- nb2listw(W.,  style="W")

#####################################
## simultaneous autoregression(SAR)
## Y = Z b + L*W (Y-Z b) + e

sar<-errorsarlm(Y ~ X1+X2+X3+X4+X5+X6+X7+X8 ,data,W)

summary(sar)
L <- sar$lambda
bh <- sar$coefficients
eh <- sar$residuals
yh<-sar$fitted
SSE <- t(eh)%*%eh
SST <- sum((Y-mean(Y))^2)
R2 <- 1-SSE/SST
S2 <- SSE/(n-r-1)

## matrice dei pesi espressa (non sparsa)
D <- matrix(0,n,n)
for(k in 1:n){ for(j in 1:4){
    D[k,d$nn[k,j] ]<-0.25}}
## W e D sono la stessa cosa...

###########################################
## regressione lineare (OLS):
## Y = Z B + EPS

OLS <- lm(Y ~ X1+X2+X3+X4+X5+X6+X7+X8)

summary(OLS)

B1 <- OLS$coefficients
e1 <- OLS$residuals
estd1 <- rstandard(OLS)
Yh1 <- OLS$fitted
hii1 <- hatvalues(OLS)


SSe1 <- t(e1)%*%e1
R2linreg <- 1-SSe1/SST
 S2.1<-SSe1/(n-r-1)

shapiro.test( estd1 )

#######################################################

windows()
 par(mfrow=c(2,2))
 plot(X1, Y, col=2, pch=16, main='X1 -OLS',cex=0.5)
 points(X1, Yh1, col=3, pch=16,cex=0.3)
 legend(10.5,11.5,c('Y','Y_OLS'),col=c(2,3),pch=16,cex=0.5)
 plot(X1,Y, col=2, pch=16, main='X1 - SAR',cex=0.5)
 points(X1, yh, col=4,pch=16,cex=0.3)
 legend(10.5,11.5,c('Y','Y_SAR'),col=c(2,4),pch=16,cex=0.5)
 plot(X2,Y, col=2, pch=16, main='X2 - OLS',cex=0.5)
 points(X2, Yh1, col=3,pch=16,cex=0.3)
 legend(140,11.5,c('Y','Y_OLS'),col=c(2,3),pch=16,cex=0.5)
 plot(X2,Y, col=2, pch=16, main='X2 - SAR',cex=0.5)
 points(X2, yh, col=4,pch=16,cex=0.3)
 legend(140,11.5,c('Y','Y_SAR'),col=c(2,4),pch=16,cex=0.5)

#######################################################

 ## PREVISIONE ##

## Xo = matrice (8,q) le cui righe sono i valori dei
## regressori sui nuovi individui sui quali si vuole effettiare le
## previsione. Per i nuovi q individui sono noti i valori assunti
## dai regressoni (Xo) e anche latitudine e longitudine.

## In base al modello:
## fitted:   yh.o = Zo bh + L*D.o(yo - Zo bh)
## Zo_i = (1,X1_oi,X2_oi,...,X8_oi)
## D.o = matrice dei pesi per i
## nuovi dati yo (valori della variabile risposta in corrispondenza
## dei nuovi individui)
## yo non � noto (� proprio quello che si
## deve stimare) => l'unica stima dei fitted che si pu� ottenere con
## le informazioni  del nuovo dataset:
## yh.o = Zo bh

## IDEA: se i nuovi dati sono 'vicini' ai dati del primo data set:
## fitted:   yh.o = Zo bh + L*D.o(Y - Z bh)
## D.o = matrice dei pesi per il nuovo data set (n,q)
## la riga i_esima di D.o contiene i 4 individui del data set
## originale pi� vicini all'i_esimo individuo del nuovo dataset
## in questo modo i nuovi dati sono supposti dipendere dai 4
##individui pi� vicini non dello stesso data set (per cui le y non
## sono note ma vanno stimate) ma del data set originale. Questo ha
## senso se gli individui di Xo sono abbastanza vicini ad almeno 4
## individui di X.

## uso q=100 dati per fare la previsione

dati.new <- dati.comp[1001:1100,]
attach(dati.new)

Yo <- log(median_house_value)

X1o <- median_income
X2o <- median_income^2
X3o <- median_income^3
X4o <- log(housing_median_age)
X5o <- log(total_rooms/population)
X6o <- log(total_bedrooms/population)
X7o <-log(population/households)
X8o <- log(households)
lato <- latitude
longo <- longitude
detach(dati.new)
q <- length(Yo)

Zo <- cbind(rep(1,q),X1o,X2o,X3o,X4o,X5o,X6o,X7o,X8o)
dati.prev <- data.frame(Zo[,2:9])
Xo <- as.matrix(dati.prev)
var <- c('X1','X2','X3','X4','X5','X6','X7','X8')
dimnames(dati.prev)[[2]]<-var

## creiamo la matrice dei pesi W per il nuovo data set Xo

## 1) valutazione per ogni individui i di Xo dei 4 individui del
## dataset iniziale X pi� vicini a i (distanza euclidea valutata
## sulle coord. lat o long.)

g <- matrix(0,q,4) for(i in 1:q)
    g[i,] <- knearneigh(cbind(c(lat,lato[i]),c(long,longo[i])), k=4)$nn[1001,]

## 2) D.o: matrice dei pesi  (q, n)
## per ogni riga i, D[i,j] = 0.25 se j � il valore corrispondente ad
## uno dei 4 'vicini' di i, altrimenti D[i,j]=0.

Do <- matrix(0,q,n)

for(k in 1:q){ for(j in 1:4){
    Do[ k, g[k,j] ] <-0.25}}

## trend = Zo bh   (� quello che si ottiene anche con predict.sarlm)
trend <- Zo%*%bh

trend predict.sarlm(sar,newdata=dati.prev,W)

## signal = L*D.o(Y - Z bh)
signal <- L*Do%*%(Y-Z%*%bh)

## yh.o = trend + signal
yh.o <- trend+signal

## previsione con Regressione Lineare:
yh1.o <-Zo%*%B1

## distanze tra gli yo (noti) e quelli stimati da i 2 modelli
delta.sar <- Yo-yh.o
delta.ols <- Yo - yh1.o

oss <- c('DATI NOTI','SAR','OLS','delta SAR','delta OLS.')
LL <-cbind(Yo,yh.o,yh1.o,delta.sar,delta.ols)
dimnames(LL)[[2]]<-oss

windows()
par(mfrow=c(2,2))

plot(X1o,Yo,col=2,cex=0.7,ylim=c(min(Yo,yh.o,yh1.o),max(Yo,yh.o,yh1.o))
,pch=16,main='Previsioni- X1',xlab='X1',ylab='Y' )
points(X1o,yh.o,col=4,pch=16,cex=0.7)
points(X1o,yh1.o,col=3,pch=16,cex=0.7)

plot(X4o,Yo,col=2,ylim=c(min(Yo,yh.o,yh1.o),max(Yo,yh.o,yh1.o)),pch=16,
main='Previsioni- X4',xlab='X1',ylab='Y' ,cex=0.7)
points(X4o,yh.o,col=4,pch=16,cex=0.7)
points(X4o,yh1.o,col=3,pch=16,cex=0.7)


alpha <- B1[2:9]
alpha[1] <- 0
plot(X1o,Yo-Xo%*%alpha,type='p',pch=16,cex=0.7,col=2,
main=paste('Proiezione su X1'),ylab="proiezione",xlab='X1')
points(X1o,yh.o-Xo%*%alpha,pch=16,col=4,cex=0.7)
points(X1o, yh1.o-Xo%*%alpha,col=3,type='l')
points(X1o, yh1.o-Xo%*%alpha,col=3,pch=16,cex=0.7)

alpha <- B1[2:9] alpha[4] <- 0
plot(X4o,Yo-Xo%*%alpha,type='p',pch=16,cex=0.7,col=2,
main=paste('Proiezione su X4'),ylab="proiezione",xlab='X2')
points(X4o,yh.o-Xo%*%alpha,pch=16,col=4,cex=0.7)
points(X4o, yh1.o-Xo%*%alpha,col=3,type='l')
points(X4o, yh1.o-Xo%*%alpha,col=3,pch=16,cex=0.7)


## dal grafico sembra che siano migliori le previsioni fatte con
## la regressione lineare che non con la sar

## ma se analiziamo le distanze tra il data set iniziale X e Xo

## vediamo in effetti quanto i punti di X e Xo siano tra loro vicini

windows()
 plot(lat,long,pch=4,cex=0.8,lwd=2, col=1,main='Coordinate dei dataset',
 xlim=c(min(c(lat,lato)),max(c(lat,lato))), ylim=c(min(c(long,longo)),
 max(c(long,longo))),xlab='latitudine',ylab='longitudine'   )
points(lato,longo,pch=4,col='orange',lwd=2,cex=0.8)
legend(37.5,-119.7,col=c('black','orange'),c('campione X','campione
Xo'),pch=4,cex=0.8)


## vediamo che in effetti che i dati di Xo sono molto distanti da X
## � quindi in accordo con il modello che la sar dia delle cattive
## previsioni

## proviamo a selezionare gli unici punti di Xo che sono vicini ad
## almeno 4 di X lato < 38.0 sono i primi 22 dati di Xo

Xo2<-Xo[1:22,]
t <- 22
lato2 <- lato[1:t]
longo2 <- longo[1:t]
points(lato2,longo2,pch=16,col=6)

windows()

plot(lat,long,pch=4,cex=0.8,lwd=2, col=1,main='Coordinate dei
dataset',xlim=c(min(c(lat,lato)),38),ylim=c(min(c(long,longo)),-121.5),
xlab='latitudine',ylab='longitudine'   )
points(lato,longo,pch=4,col='orange',lwd=2,cex=0.8)
legend(37.88,-121.49,col=c('black','orange'),c('campione
X','campione Xo'),pch=4,cex=0.8)

## vediamo quanto i valori di y predetti dalla sar sono vicini a
## quelli veri

LL[1:t,]

windows()
par(mfrow=c(2,2))
plot(X1o[1:t],Yo[1:t],col=2,cex=1.2,ylim=c(11.8,max(Yo,yh.o,yh1.o)),
pch=16,main='Previsioni- X1',xlab='X1',ylab='Y' )
points(X1o[1:t],yh.o[1:t], col=4,pch=16,cex=1.2)
points(X1o[1:t],yh1.o[1:t], col=3,pch=16,cex=1.2)

plot(X4o[1:t],Yo[1:t],col=2,cex=1.2,ylim=c(11.8,max(Yo,yh.o,yh1.o)),
pch=16,main='Previsioni - X4',xlab='X4',ylab='Y' )
points(X4o[1:t],yh.o[1:t], col=4,pch=16,cex=1.2)
points(X4o[1:t],yh1.o[1:t], col=3,pch=16,cex=1.2)

alpha <- B1[2:9] alpha[1] <- 0
plot(X1o[1:t],Yo[1:t]-Xo2%*%alpha,type='p',pch=16,cex=1.2,col=2,
main=paste('Proiezione su X1'),ylab="proiezione",xlab='X1')
points(X1o[1:t],yh.o[1:t]-Xo2%*%alpha,pch=16,col=4,cex=1.2)
points(X1o[1:t], yh1.o[1:t]-Xo2%*%alpha,col=3,type='l')
points(X1o[1:t], yh1.o[1:t]-Xo2%*%alpha,col=3,pch=16,cex=1.2)

alpha <- B1[2:9] alpha[4] <- 0
plot(X4o[1:t],Yo[1:t]-Xo2%*%alpha,type='p',pch=16,cex=1.2,col=2,
main=paste('Proiezione su X4'),ylab="proiezione",xlab='X4')
points(X4o[1:t],yh.o[1:t]-Xo2%*%alpha,pch=16,col=4,cex=1.2)
points(X4o[1:t], yh1.o[1:t]-Xo2%*%alpha,col=3,type='l')
points(X4o[1:t], yh1.o[1:t]-Xo2%*%alpha,col=3,pch=16,cex=1.2)


###################################################
###################################################

## altri grafici:

## rosso = dati reali
## verde = regressione lineare
## blue = sar

## regressione lineare OLS
## 1)
## projection of fitted and response
vs regressors plot

windows()
layout(matrix(1:8,2,4,byrow=T))

for(i in 1:p){
    alpha <- B1[2:9]
    alpha[i] <- 0
    plot(X[,i],Y-X%*%alpha,type='p',pch=16,col=3,main=
    paste('Proiezione su',names(data)[i+1]),ylab="proiezione",
    xlab=names(data)[i+1])
    points(X[,i],Yh1-X%*%alpha, type='l',pch=16,col=2)
}

## 2)
## studentized residuals vs fitted plot
windows()
plot(Yh1,estd1,type='p',pch=16,col=3,main='Residui studentizzati vs
Fitted',ylab="residui studentizzati",xlab="fitted")

## 3)
## studentized residuals vs regressors plot
windows()
layout(matrix(1:8,2,4,byrow=T)) for(i in 1:p){
    plot(X[,i],estd1,type='p',pch=16,col=3,main=paste('Residui studentizzati
    vs',names(data)[i+1]),ylab="residui studentizzati",xlab=names(data)[i+1])
}

## 4)
## leverage vs fitted plot
avarage.lavarage <- (p+1)/n
windows()
plot(Yh1,hii1,type='p',pch=16,col=3,main='Leverage vs
Fitted',ylab="leverage",xlab="fitted")
abline(avarage.lavarage,0,col=2)

## 5)
## studentized residuals qqplot
windows()
qqnorm(estd1, main='Residui studentizzati QQplot',col=3, pch=16)
qqline(estd1,col=2)

## SAR

windows()
par(mfrow=c(2,2))
for(i in 1:4){
    alpha <- B1[2:9]
    alpha[i] <- 0
    plot(X[,i],Y-X%*%alpha,type='p',pch=16,cex=0.5,col=2,main=
    paste('Proiezione su',names(data)[i+1]),ylab="proiezione",
    xlab=names(data)[i+1])
    points(X[,i],yh-X%*%alpha,pch=16,col=4,cex=0.3)
    points(X[,i], Yh1-X%*%alpha,col=3,type='l')
      points(X[,i], Yh1-X%*%alpha,col=3,pch=16,cex=0.3)
} windows() par(mfrow=c(2,2)) for(i in 5:p){
    alpha <- B1[2:9]
    alpha[i] <- 0
    plot(X[,i],Y-X%*%alpha,type='p',pch=16,cex=0.5,col=2,main=
    paste('Proiezione su',names(data)[i+1]),ylab="proiezione",
    xlab=names(data)[i+1])
    points(X[,i],yh-X%*%alpha,pch=16,col=4,cex=0.3)
    points(X[,i], Yh1-X%*%alpha,col=3,type='l')
    points(X[,i], Yh1-X%*%alpha,col=3,pch=16,cex=0.3)
}



## PREVISIONI

windows()
layout(matrix(1:8,2,4,byrow=T))
for(i in 1:p){
    alpha <- B1[2:9]
    alpha[i] <- 0
    plot(X[,i], Yh1-X%*%alpha,col=3,type='l',main=
    paste('Proiezione su',names(data)[i+1]),ylab="proiezione",
    xlab=names(data)[i+1])
    points(Xo2[,i],Yo[1:22]-Xo2%*%alpha,col=2,pch=16)
    points(Xo2[,i],yh.o[1:22]-Xo2%*%alpha,col=4,pch=16)
    points(Xo2[,i],yh1.o[1:22]-Xo2%*%alpha,col=3,pch=16)
}

windows()
layout(matrix(1:8,2,4,byrow=T))
for(i in 1:p){
    alpha <- B1[2:9]
    alpha[i] <- 0
    plot(X[,i], Yh1-X%*%alpha,col=3,type='l',main=
    paste('Proiezione su',names(data)[i+1]),ylab="proiezione",
    xlab=names(data)[i+1])
    points(Xo[,i],Yo-Xo%*%alpha,col=2,pch=16)
    points(Xo[,i],yh.o-Xo%*%alpha,col=4,pch=16)
    points(Xo[,i],yh1.o-Xo%*%alpha,col=3,pch=16)
}




\end{verbatim}
